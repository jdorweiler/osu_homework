\documentclass{article}
\usepackage{fullpage}	
\setlength\parindent{0pt}
\everymath{\displaystyle}

\begin{document}
Jason Dorweiler

Assignment 6.3

Section 10.5: 10, 14, 26a,b,c (7th edition)\\

10. Can someone cross all the bridges shown in this map exactly once and return to the
starting point?\\

Yes this graph has an even number of vertices so by definition has a Euler circuit. The example of crossing all bridges once being A, C, B, D, C, B, A. \\

14. Determine whether the picture shown can be drawn with a pencil in a continuous motion
without lifting the pencil or retracing part of the picture?\\

Yes, adding a vertex at each coroner, intersection of two lines, and two on the ends of the graph.  Each node will have an even number of edges except for the two on the ends.  The two on the ends are odd and there are only two odd nodes.  So by definition there is a path we can draw without lifting a the pencil once. \\

26. For which values of n do these graphs have an Euler circuit?\\

a. $K_n$ This graph will have a Euler circuit when there are an odd number of vertices. 

b. $C_n$  This graph will have an odd number for $n\geq3$

c. $W_n$ These graphs never have a Euler circuit since each node always has an odd number of edges. 
\end{document}

