\documentclass[12pt]{article}
\usepackage{fullpage}
\title{CS225 Midterm Review}
\setlength\parindent{0pt}
\author{Jason}
\begin{document}
\maketitle

Section 1.7\\

1. Use a direct proof to show that the sum of two odd integers
is even.

$n_1+n_2 = odd$ when n is an integer.

$n_1 = 2k_1+1, n_2 = 2k_2+1$

$2k_1+1+2k_2+1$

$2*(k_1+k_2+1)$ def of even number. \\

3. Show that the square of an even number is an even number
using a direct proof.

$n^2$ is even.

$n = 2k$ 

$(2k)^2 = 2*(2k^2)$ definition of even number\\

6. Use a direct proof to show that the product of two odd
numbers is odd.

$(2k_1+1)*(2k_2+1)$

$4*k_1k_2+2k_1+2k_2+1$

$2(2k_1k_2+k_1+k_2) +1$ definition of odd number.\\

14. Prove that if x is rational and x = 0, then 1/x is rational.

x is rational so $x= \frac{a}{b}$, where a and b are integers. 

$\displaystyle{\frac{1}{\frac{a}{b}} = \frac{b}{a}}$ which is also rational.  \\

18. Prove that if n is an integer and 3n + 2 is even, then n is
even using

a) a proof by contraposition.

If n is odd, then $3n+2$ is odd.

n = 2k+1

$3*(2k+1)+2 = 6k+5 = 2*(3k+2)+1$\\

b) a proof by contradiction.

Assume for contradiction that n is odd, $n=2k+1$.

$3*(2k+1)+2 = 6k+5 = 2*(3k+2)+1$ which says that $3*n+2$ is odd which is a contradiction. \\

8. Prove that if n is a perfect square, then n + 2 is not a
perfect square.

Using proof by contradicion. Assume that n+2 is a perfect square.  

$n+2 = k*k$

$n = kk+2$ so n is not a perfect square.  This is a contradiction to our assumption. \\

\hrulefill\\

\textbf{Set Notation}\\

Sets are unordered collections of objects. \\

O is the set of all odd positive integers less than 10. 

$O = \{x \in Z^+ | x\,is\,odd\,and\,x < 10\}$ \\

Two sets are equal if and only if they have the same elements. Therefore, if A and B are sets,
then A and B are equal if and only if $\forall x (x \in A \leftrightarrow x \in B)$. We write A = B if A and B are
equal sets.\\

The set A is a subset of B if and only if every element of A is also an element of B. We use
the notation $A \subseteq B$ to indicate that A is a subset of the set B.\\

Theorem 1 shows that every nonempty set S is guaranteed to have at least two subsets, the
empty set and the set S itself, that is, $ \emptyset \subseteq S$ and $S \subseteq S$

\hrulefill\\

\textbf{Set Operations}\\

16. Let a and b be sets. show:

a. $(A \cap B) \subseteq A$

suppose $x \in (A \cap B)$ then $x \in A, x \in B$ since $x \in A$ we can conclude that it's true. \\

b. $A \subseteq (A \cup B)$

Suppose that $x \in A$ then it follows that $x \in A$ or $x \in B$ and so it's true.

c. $A - B \subseteq A$

suppose that $x \in (A - B)$ then it follows that $x \in A$ and $x \notin B$ 

d. $A \cap (B-A) = \emptyset$

Proof by contradiction. Assume that $A \cap (B-A) \neq \emptyset$ and  $x \in A \cap (B-A)$ it follows that $x \in A$ and $x \in B, x \notin A$ which is a contradiction. 

e. $A \cup (B-A) = A \cup B$

\begin{tabular}{|c|c|c|c|c|c|}
\hline 
A & B & $\neg A$ & (B-A) & $A \cup (B-A)$ & $A \cup B$ \\ 
\hline 
1 & 1 & 0 & 0 & 1 & 1 \\ 
\hline 
1 & 0 & 0 & 0 & 1 & 1 \\ 
\hline 
0 & 1 & 1 & 1 & 1 & 1 \\ 
\hline 
0 & 0 & 1 & 0 & 0 & 0 \\ 
\hline
\end{tabular}\\


18. a. $(A \cup B) \subseteq (A \cup B \cup C)$

These are all pretty much the same. 

1. assume x is an element of the left hand side. Break it down into each element and come to a conclusion. 

2. if it proving something is NULL then use contradiciton. 

3. if something  =  something else just use a table. \\

\textbf{Quiz 4 questions}\\

4. Let A: the set of all red cars

B: the set of all fast cars. 

The set of all caars that are neither red nor fast. 

$\neg A \cap \neg B$\\


5. Prove: $A - B \subseteq \overline{B}$

Suppose that $x \in (A-B)$. Then $x\in A$ and $x \notin B$. Since $x \notin B$ it follows that  $A - B \subseteq \overline{B}$ is true.\\

\hrulefill

\textbf{Sums and Sequences}\\

29.\\
$\sum_{k=1}^{5} (k+1) = 1+1 + 2+1 + 3+1 +4+1 + 5+1 = 20$\\

$\sum_{j=0}^{4} (-2)^j = \frac{(-2)^{4+1}-1}{(-2)-1}$\\

$\sum_{i=1}^{10} 3 = 3*10 = 30$\\

$\sum_{j=0}^{8} (2^{j+1}-2^j) = \sum_{j=0}^{8} 2^{j} = \frac{(2)^{8+1}-1}{(2)-1}$\\

31.\\
$\sum_{j=0}^{8} 3*2^j = 3*\frac{(2)^{8+1}-1}{(2)-1}$\\

$\sum_{j=1}^{8} 2^j = \frac{(2)^{8+1}-1}{(2)-1} - 1$\\

$\sum_{j=2}^8 (-3)^j = \frac{(-3)^{8+1}-1}{(-3)-1} - \frac{(-3)^{1+1}-1}{(-3)-1} $\\

$\sum_{j=0}^8 2 \cdot(-3)^j = 2* \frac{(-3)^{8+1}-1}{(-3)-1}$\\

32.\\
$\sum_{j=0}^8 (1+(-1)^j) = \sum_{j=0}^8 1+\sum_{j=0}^8(-1)^j$\\

\hrulefill\\

\textbf{Weak Induction}\\

General method to complete proofs:

Step 1: Show that the basis P(0) or P(1) is true.  Sometimes you need to read the problem to see what the basis is.  It is not always 0 or 1. 

Step 2: Substitute k in for n or whatever variable there is. On both sides of the equation. 

Step 3: add k+1 to the right side. 

Step 4:  Do math and show that it is equal to step 2.\\

Ex:

$1*1!+2*2!....n*n! = (n+1)!-1$

Basis: 0*0 = 1-1 = 0

Inductive:

$k*k! = (k+1)! - 1$

$k*k!+(k+1)*(k+1)! = (k+1+1)! -1 + (k+1)*(k+1)!$
  
\end{document}