\documentclass[12]{article}

\begin{document}
12.


Basis Step:

$f_1^2 = f_1*f_2$ 

$f_1 = 1$ and $f_2 = 1$ 

$f_1^2 = 1*1 = f_1 * f_2$ so the basis step checks out.\\

Recursive Step:

Assume that $f_1^2+f_2^2+\cdot\cdot\cdot+f_n^2=f_n*f_{n+1}$

Prove: $f_1^2+f_2^2+\cdot\cdot\cdot+f_n^2+f_{n+1}=f_n*f_{n+1}*f_{n+2}$

Rewriting the assumption:

$(f_1^2+f_2^2+\cdot\cdot\cdot+f_n^2)+f_{n+1}^2$ where $(f_1^2+f_2^2+\cdot\cdot\cdot+f_n^2)$ is our hypothesis.  

We can rewrite as $f_n*f_{n+1}+f_{n+1}^2$ and factor to get $f_{n+1}*(f_n+f_{n+1})$

which is the definition of a Fibonacci number $f_{n+1}f_{n+2}$ and proves our 

hypothesis. \\




26 c)

Basis step:

The basis $\{0,0\}$ is still valid since $5|\{0,0\}=0$\\

Recursive Step:
Suppose that $a+b=5k$.  Then $5|\{a+2,b+3\}$ 

which is $a+b+5 = 5k+5 = 5(k+1)$ where $k+1$ is also an integer.

The same is true for $5|\{a+3,b+2\}$ which is $a+b+5 =5k+5 = 5(k+1)$

This completes the proof.\\


43.

Base Case:

$n(T) = 1$ then the height is $h(T) = 0$ and $1 \geq 2*0+1$

Assume that the hypothesis holds for trees smaller then T.  We will make a 

new tree from subtrees $T_1$ and $T_2$.  Since the hypothesis is still true we know 

that $n(T_1) \geq 2*h(T_1)+1$ and the same applies for $T_2$.  The new tree is then 

$n(T_n) = 1 + n(T_1)+(T_2)$ accounting for the top node.  

Rewriting is equal to:

$n(T_n) = 1 + 2*h(T_1)+1+2*h(T_2)+1$

\indent \indent $1 + 2(h(T_1)+h(T_2)+1)$

\indent  \indent $1 + 2*h(T)$ the height of both trees+1 is the height of the larger tree.

This completes the proof.\\
\newpage
44.

Base Case:

$n=1$ so $l(T) = 1$ since a single node tree has no internal vertices the base 

case is true. 

Recursive Case:

Assuming that the hypothesis holds for smaller trees we will make a new 

tree out of two smaller trees $T_1$ and $T_2$.  The number of leaves in the new 

tree is:
$l(T) = l(T_1) +l(T_2) $.

Substituting the definition of $i(T)$ gives:

$l(T) = i(T_1)+1 + i(T_2) +1$ 

\indent \indent i(T)+1 which is the definition of a full 

binary tree. 
This completes the proof.\\

58.

Basis Case:
The empty string $\lambda$ has the same number of left and right 

parentheses so the base case is true. \\

Recursive Case:

Assume: $x,y \in B$

Let the number of left parentheses for x,y be $l_x$ and $l_y$, and the number of 

right parentheses be $r_x$ and $r_y$.  The number of left and right parentheses for 

$(x) \in B$ will be one greater on each side. $l_x+1$ and $r_x+1$.  Since both $l_x$ 

and $r_x$ are still equal the hypothesis is true for this case.

For the $xy \in B$ case: the number of left parentheses will be $l_x+l_y$ and the 

number of right parentheses will be $r_x+r_y$  by the hypothesis we know that 

$l_x+l_y = r_x+r_y$ and so there will be the same number of parentheses. 


This completes the proof. 







 
\end{document}