\documentclass[12]{article}
\usepackage{fullpage}	
\setlength\parindent{0pt}

\begin{document}
Jason Dorweiler

Assignment 5.1 CS225

Assignment: Section 5.1: 8, 12, 16, 24, 26, 42, 46, 60 (6th edition))\\

\hrulefill\\

8. How many different three-letter initials with none of the letters repeated can people have?\\

With no repeated letters there will be $26*25*24 = 15600$ different three letter initials. \\

\hrulefill\\

12. How many bit strings are there of length six or less?\\

There will be 6 separate cases where each case is $2^{n}$. The total will be $\displaystyle{\sum_{n=0}^{6} 2^n} = 127$\\

\hrulefill\\

16. How many strings are there of four lowercase letters that have the letter
x
in the term? \\

If there were no X's then the highest number would be $26*26*26*26 = 456976$.  Since there needs to be an at least one X there will be 4 cases:\\

Case 1 (one x): With one X there are 4 combinations of where the letter can be placed in the string.  That makes the total $4*25*25*25 = 62500$\\

Case 2 (two x): There will now be 6 combinations of placements for the letter in the string but it also takes up two positions.  That makes the total $6*25*25 = 3750$\\

Case 3 (three x): There are 4 combinations of placements for three x's.  That makes the total $2*25 = 100$\\

Case 4 (four x): There is only one string with 4 x's.\\

Adding all of the cases up gives a total of $66351$ four letter strings. \\

\hrulefill\\

24. How many strings of four decimal digits

a) do not contain the same digit twice?

b) end with an even digit?

c) have exactly three digits that are 9s?\\

A: Since the string can't contain the same digit twice it can have 10 choices for the first digit, 9 for the second, 8 for the third, and 7 for the last.  Giving a total of $10*9*8*7 = 5040 $\\

B: Half of the strings will end in an even digit. The total of all four digit strings is $10^4$ and the even strings will be $\frac{10^4}{2} = 5000$\\

C: There are 4 combinations of three digits (9) in a 4 digit string.  That makes the total $4*9 = 36$\\

\hrulefill\\

26. How many license plates can be made using either three digits
followed by three letters
or three letters followed by three digits?\\

There are $10*10*10$ number combinations and $26*26*26$ letter combinations. The total for one type of plate is $10^3*26^3$.  The other plate will give the same number so the overall total is $2*(10^3*26^3) = 35152000$\\

\hrulefill\\

42. How many bit strings of length seven either begin with two 0s
or end with three 1s?\\

There will be $2^5$ strings that begin with two 0s and $2^4$ strings that end with three 1s.  The overlap between the two sets is $2^2$ so the total is $(2^5+2^4)-2^2=44$.\\

\hrulefill\\

46. Every student in a discrete mathematics class is either a computer science or a mathematics major or is a joint major in these two subjects. How many students are in the
class if there are 38 computer science majors(including joint majors), 23 mathematics majors(including joint majors), and 7 joint majors?\\


There are $38 comp sci + 23 math - 7 joint = 54$ students in the class.\\

\hrulefill\\

60. Use mathematical induction to prove the product rule for m tasks from the product rule
for two tasks.\\

$P(m)$ is the product rule for m tasks.

Basis step:

m = 2, $P(2)$ is true since there are $n_1$ ways to do the first task and $n_2$ ways to do the second one.  The total being $n_1n_2$.\\

Inductive Step:

Prove P(k) is true for $k \geq 2$.

For a k+1 tasks we have $T_1T_2 \cdot\cdot T_{k+1}$.  Each of which can be done $n_1n_2 \cdot\cdot n_{k+1}$ ways. That means that for the next k+1 task we will have $(n_1n_2 \cdot\cdot n_{k})*n_{k+1}$ ways which proves $P(k+1)$.



\end{document}