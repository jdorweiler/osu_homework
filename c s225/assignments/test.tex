\documentclass[11pt]{article}
\usepackage{fullpage}

\begin{document}

question 3. 

$ A - B \subseteq A$

Assume that $x \in A-B$. Then it follows that $x \in A$ and $x \notin B$. Since $x \in A$

 the statement $ A - B \subseteq A$ holds.\\ 

Question 4

a. All good weight lifters are strong.

$ (W(x) \rightarrow S(x))$

b) Good weight lifters are not a good basketball player

$ W(x) \rightarrow \neg B(x)$\\

c) If someone is strong then they are a good basketball player or a good weight lifter.

$ \forall x (S(x) \rightarrow (B(x) \lor W(x))$\\

d) There is someone who is a good basketball player and a good weight lifter.

$ \exists x (B(x) \land W(x))$
\\

question 5\\

\begin{tabular}{|c|c|c|c|c|c|c|}
\hline 
p & q & r & $p \rightarrow q$ (2) & $ (2) \rightarrow R$ & $P \land \neg Q$ (3) & (3) $\lor R$ \\ 
\hline 
1 & 1 & 1 & 1 & 1 & 0 & 1 \\ 
\hline 
1 & 1 & 0 & 1 & 0 & 0 & 0 \\ 
\hline 
1 & 0 & 1 & 0 & 1 & 1 & 1 \\ 
\hline 
0 & 0 & 0 & 0 & 0 & 0 & 0 \\ 
\hline 
0 & 1 & 1 & 1 & 1 & 0 & 1 \\ 
\hline 
0 & 1 & 0 & 1 & 0 & 0 & 0 \\ 
\hline 
• & • & • & • & • & • & • \\ 
\hline 
• & • & • & • & • & • & • \\ 
\hline 
\end{tabular} 

Columns 5 and 7 are identical so they are logically equivalent. 
\\

question 6

Use a direct proof to show that the sum of two rational numbers is rational. (Recall that a number is rational if and only if it can be expressed as the ratio of integers.)\\

Assume that a and b are rational numbers.  so $a = \frac{x}{z}, and b = \frac{j}{k}$ and x,z,j, and k are also integers.  Then the sum of $a+b = \frac{x+j}{z+k} $ which is also a rational number. \\

Question 7. 

Assume for contradiction that n and m have the same parity.  If m and n are even:

$n = 2k_1, m = 2k_2$

$2k_1-5(2k_2) = 2(k_1 - 5k_2)$ which is an even number and contradicts our assumption that$n-5m$ is odd. \\

For the case when they are odd.

$n=2k_1+1, m=2k_2+1$

$2k_1+1-5(2k_2+1) = 2(k_1-5k_2+3)$ which is also even and contradicts our assumption. \\

Question 8:

$\sum_{j=3}^6 (5j^2+(-1)^j) = \sum_{j=3}^6 (5j^2) + \sum_{j=3}^6 (-1)^j = (15^2+20^2+30^2)+ \sum_{j=0}^6 (-1)^j - \sum_{j=0}^2 (-1)^j $

$ = (15^2+20^2+30^2)+ (\frac{(-1)^7-1}{-1-1})- (\frac{(-1)^3-1}{-1-1})$\\

$\sum_{i=0}^8 (2^{i+1} - 3) = \sum_{i=0}^8 2^{i+1} - 8*3 = 2* \sum_{i=0}^8 2^i - 8*3$

$= 2*\frac{2^9-1}{2-1} - 24$\\

question 9:
Base case: P(0) $3^0$ = 1, $\frac{3^0-1}{2}=1$

inductive case:

$ 3^n = \frac{3^{n+1}-1}{2} $

for n=k:

$ 3^k = \frac{3^{k+1}-1}{2} $

and for n=k+1

$ 3^{k+1} = \frac{3^{k+2}-1}{2} $

assuming that $ 3^k = \frac{3^{k+1}-1}{2} $ is true then:

$\frac{3^{k+1}-1}{2} + 3^{k+1}$ = $\frac{3^{k+1}-1}{2} + \frac{2*3^{k+1}}{2} =  \frac{3^{k+1}-1+2*3^{k+1}}{2} = \frac{3^{k+1}*(1+2)-1}{2} = \frac{3^{k+2}-1}{2}$ so the hypothesis holds. \\

question 10:

Use strong induction to prove that every amount of postage of 12 cents or more can be formed using just 4-cent and 5-cent stamps\\

Base case:

 P(12) = 4*4*4 = 12 

p(13) = 4*4*5 = 13

p(14) = 5*5*4 =14

This proves the first three base cases. n, n+1, and n+2 

Inductive case: 

Assume a new stamp k such that $12 \leq k \leq j$. We know p(n)...p(n+2) is true. The next stamp P(n+3) has to be true since P(n+2) was proven in our base case.  

\end{document}