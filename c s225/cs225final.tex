\documentclass{article}
\usepackage{fullpage}	
\setlength\parindent{0pt}
\everymath{\displaystyle}


\begin{document}
CS225 final. Jason Dorweiler 


\textbf{question 1}

\begin{tabular}{|c|c|c|c|c|c|c|c|c|c|}
\hline 
$\neg P$ & P & Q & R & $\neg Q$ & $\neg Q \land R$ (1) & $P \rightarrow 1$ & $R \rightarrow Q$ (2) & $\neg 2$ (3) & $\neg P \lor 3$ \\ 
\hline 
0 & 1 & 1 & 1 & 0 & 0 & 0 & 1 & 0 & 0 \\ 
\hline 
0 & 1 & 1 & 0 & 0 & 0 & 0 & 1 & 0 & 0 \\ 
\hline 
0 & 1 & 0 & 1 & 1 & 1 & 1 & 0 & 1 & 1 \\ 
\hline 
0 & 1 & 0 & 0 & 1 & 0 & 0 & 1 & 0 & 0 \\ 
\hline 
1 & 0 & 1 & 1 & 0 & 0 & 1 & 1 & 0 & 1 \\ 
\hline 
1 & 0 & 1 & 0 & 0 & 0 & 1 & 1 & 0 & 1 \\ 
\hline 
1 & 0 & 0 & 1 & 1 & 1 & 1 & 0 & 1 & 1 \\ 
\hline 
1 & 0 & 0 & 0 & 1 & 0 & 1 & 1 & 0 & 1 \\ 
\hline 
\end{tabular} \\

Columns 7 and 10 are identical so the two expressions are logically equivalent\\


\textbf{question 2}

If $x^2+2x^3=5$ then $x < 2$ is the expression $P \rightarrow Q$.  For proof by contropositive I will show $\neg Q \rightarrow \neg P$ that is if $x \geq 2$ then $x^2+2x^3\neq5$.  Assume that x = 2 then $2^2+2(2)^3 = 24$ which proves the expression $x^2+2x^3\neq5$ and $\neg Q \rightarrow \neg P$ this proves $P \rightarrow Q$\\

\textbf{Question 3}

Basis: P(2): 2 $<$ 4

Inductive: $n! < n^n$, assume n = k, then $k! < k^k$

$k!= k(k+1) < (k+1)^{(k+1)} < (k+1)^{k+1}(k+1)$.  This shows that the next k+1 in each series for the right hand size is still greater than that of the left hand side of the equation.

\textbf{question 4}

For the left hand side assume that $x \in (A-B)-C$, then it follow that $x\in A, x\notin B, x \notin C$

For the right hand side assume that $x \in A-C$ then it follows that $x \in A, x \notin C$.  so from LHS $x \in A$ and from the RHS $x \in A$. This completes the proof. 

\textbf{question 5} 

$a_n = n^2 - 2$ 

$a_{n+1} = (n+1)^2-2 = n^2+2n+1-2 = n^2-2+2n+1 = a_n +2n+1$

\textbf{question 6}

Basis is given so that part is already done.

For the inductive step I will show the case of P(1,1): 

a = 1+0, b = 1+1, $1 \leq 4$

a = 1+1, b = 1+1  $ 2 \leq4$

a = 1+2, b = 1+1 $ 3 \leq4$

For all three cases it holds that $a \leq 2b$.  For the case of P(2,2):

a = 2+0, b = 2+1 $ 2 \leq6$

a = 2+1, b = 2+1 $ 3 \leq6$

a = 2+2, b = 2+1 $ 4 \leq6$

For all of these cases it still holds that $a \leq 2b$ so this proves that $P(1,1) \rightarrow P(2,2)$ and then $P(n-1,n-1) \cdots P(n,n)$ must be true. 

\textbf{question 7}

a. $36CS + 21Math - 7joint = 50$

b. $2^6+2^5+2^4+2^3+2^2+2^1+2^0$


\textbf{question 8}

a. C(14,12) = $\frac{14!}{12!(2!)}$

b.total number of combinations C(15,9) = $\frac{15!}{9!(6!}$. Number of ways to a team with only men C(12,9) = $\frac{12!}{9!(3!)}$.  So the number of teams with at least on woman is the total teams - number with only men = $\frac{15!}{9!(6!} - \frac{12!}{9!(3!)}$

\textbf{question 9}

In order for the graph to be a Euler cycle we must find a path from our starting vertex through each edge and land back where we started. This means that we must have both an entering and exiting edge to each node.  This implies that the degree of the node has to be a multiple of 2.  This shows that a Euler cycle needs to have an even degree. \\   

\textbf{question 10 }

\begin{tabular}{|c|c|c|c|c|c|c|}
\hline 
Start A, end Z & B & C & D & E & Z & iteration - added to set \\ 
\hline 
• & 2 & 3 & - & - & - & 1 - A \\ 
\hline 
• & 2 & 3 & 7 & 4 & - & 2 -B \\ 
\hline 
• & 2 & 3 & 7 & 4 & - & 3 - C \\ 
\hline 
• & 2 & 3 & 7 & 4 & 8 (via e,b) & 4 - E \\ 
\hline 
• & 2 & 3 & 7 & 4 & 8 & 5 - D \\ 
\hline 
\end{tabular} 

Then we reach Z on iteration 6. 
The shortest path is A,B,E,Z
\end{document}



