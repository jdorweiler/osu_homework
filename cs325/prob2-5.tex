\documentclass[10pt,a4paper]{article}
\usepackage[latin1]{inputenc}
\usepackage{amsmath}
\usepackage{amsfonts}
\usepackage{amssymb}
\begin{document}

Using the three cases for the ratio: $$(\frac{a}{a^d})$$
from DPV pg. 60:\\
Case 1: $
(\frac{a}{a^d}) < 1
$
Then the series is decreasing.  It's sum is given by $O(n^d)$\\
Case 2: $
(\frac{a}{a^d}) > 1
$
Then the series is decreasing.  It's sum is given by $n^{\log_b^a}$\\
Case 3: $
(\frac{a}{a^d}) = 1
$
Then the series is decreasing.  It's sum is given by $n^d\log(n)$\\

(a)
$T(n) = 2T(\frac{n}{3})+1$

case 1: $\Theta(n)$\\

(b)
$T(n) = 5T(\frac{n}{4})+n$

case 2: $\Theta(n^{\log_4^5})$\\

(c)
$T(n) = 7T(\frac{n}{7})+n$

case 3 with d = 1: $\Theta(n\log(n))$\\

(d)
$T(n) = 9T(\frac{n}{3})+n^2$

case 3 with d = 2: $\Theta(n^2\log(n))$\\

(e)
$T(n) = 8T(\frac{n}{2})+n^3$

case 3 with d = 3: $\Theta(n^3\log(n))$\\

(f)
$T(n) = 49T(\frac{n}{25})+n^{\frac{3}{2}}\log(n)$

case 2 but the non-recursive portion dominates so: $\Theta(n^{\frac{3}{2}}\log(n))$\\

(g)
$T(n) = T(n-1)+2$

The depth of the recursion tree will be n so it's easy to see that this one is $\Theta(n)$\\

(h)
$T(n) = T(n-1)+n^c$

In this case the depth of the recursion tree is still n but each level of the tree takes $n^c$ so: $\Theta(n*n^c)$\\

(i)
$T(n) = T(n-1)+c^n$

In this case the depth of the recursion tree is still n but $c^n$ for c $>$ 1 is going to dominate so: $\Theta(c^n)$\\

(j)
$T(n) = 2T(n-1)+1$

The depth of the tree will be n and time 2 at each level: $\Theta(2^n)$\\

(k)
$T(n) = T(\sqrt{n})+1$
Stuck on this one. There's going to be a $\log(n)$ in there somewhere since n decreases by $\sqrt{n}$ at each level. 
: $\Theta(?)$\\
\end{document}